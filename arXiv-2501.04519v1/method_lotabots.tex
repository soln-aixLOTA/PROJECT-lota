\section{System Architecture and Implementation}

\subsection{Core Technologies}

\sysname{} is built on a modern technology stack:

\begin{itemize}
    \item \textbf{Language}: Rust (Edition 2021) for performance and safety
    \item \textbf{Web Framework}: Actix-web 4.4 for async web services
    \item \textbf{Database}: PostgreSQL 14.2+ with JSON support
    \item \textbf{Caching}: Redis 6.2+ for session management
    \item \textbf{Runtime}: Tokio 1.35+ for async I/O
\end{itemize}

\subsection{File Structure}

The project follows a well-organized directory structure:

\begin{verbatim}
lotabots/
  services/             # Microservices
    api-gateway/        # Entry point for external clients
    auth/               # Authentication logic
    attestation/        # Hardware & AI model verification
    document/           # Document management
    resource-management/# Resource allocation and monitoring
  shared/               # Shared libraries
    config/             # Configuration handling
    db/                 # Database utilities
    models/             # Shared data structures
    utils/              # Common utilities
  infrastructure/       # Infrastructure as code (Terraform, etc.)
  scripts/              # Development scripts
  docs/                 # Documentation
  tests/                # Integration tests
  tools/                # Development tools
\end{verbatim}

\subsection{Service Architecture}

Each microservice follows a consistent architecture:

\begin{itemize}
    \item \textbf{Handlers}: Process incoming HTTP requests
    \item \textbf{Models}: Define data structures and validation
    \item \textbf{Repository}: Handle database operations
    \item \textbf{Services}: Implement business logic
    \item \textbf{Config}: Manage service configuration
\end{itemize}

\subsection{Security Implementation}

Security measures are implemented at multiple levels:

\begin{itemize}
    \item \textbf{Authentication}: JWT-based with refresh tokens
    \item \textbf{Password Security}: bcrypt with configurable work factor
    \item \textbf{API Security}: Rate limiting, CORS, CSP headers
    \item \textbf{Secrets}: HashiCorp Vault or AWS Secrets Manager
    \item \textbf{Network}: TLS 1.3, mTLS for service communication
\end{itemize}

\subsection{Deployment Architecture}

The system is designed for cloud-native deployment:

\begin{itemize}
    \item \textbf{Containerization}: Multi-stage Docker builds
    \item \textbf{Orchestration}: Kubernetes with Helm charts
    \item \textbf{Infrastructure}: AWS (EKS, RDS, ElastiCache)
    \item \textbf{CI/CD}: GitHub Actions with security scans
    \item \textbf{Monitoring}: Prometheus, Grafana, ELK Stack
\end{itemize}

\subsection{Development Workflow}

The development process follows best practices:

\begin{itemize}
    \item \textbf{Code Quality}: cargo fmt, clippy, audit
    \item \textbf{Testing}: Unit, integration, and e2e tests
    \item \textbf{Documentation}: OpenAPI specs, inline docs
    \item \textbf{Review}: PR reviews with automated checks
    \item \textbf{Version Control}: Git with conventional commits
\end{itemize}
