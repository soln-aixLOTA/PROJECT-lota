\section{Related Works}

\noindent\textbf{Math Data Synthesis}.  Advancements in LLM math reasoning have largely relied on curating high-quality  CoT data, with most leading approaches being GPT-distilled, using frontier models like GPT-4 for synthesis~\citep{wang2024mathcoder,tora,luo2023wizardmath}. Notable works include NuminaMath~\citep{numina_math_datasets} and
MetaMath~\citep{yu2023metamath}. While effective, this limits reasoning to the capabilities of the teacher LLM.  
Hard problems that the teacher LLM cannot solve are excluded in the training set. 
   Even solvable problems may contain error-prone intermediate steps, which are hard to detect. Although rejection sampling methods~\citep{rejectionsampling,brown2024large} can improve data quality,
 they do not guarantee correct intermediate steps. As a result, scaling up CoT data has diminishing returns, with gains nearing saturation—e.g., OpenMathInstruct-2~\citep{openmath2} only sees a 3.9\% boost on MATH despite an 8× increase in dataset size.
   
   \noindent\textbf{Scaling Test-time Compute} has introduced new scaling laws,  allowing LLMs to improve performance across by generating multiple samples and using reward models for best-solution selection ~\citep{snell2024scaling,wu2024empirical,brown2024large}. Various test-time search methods have been proposed~\citep{ kang2024mindstar,wang2024qimprovingmultistepreasoning}, including  random sampling~\citep{selfconsistency} and tree-search methods~\citep{tot,rap,zhang2024accessing,rstar} like MCTS. However, open-source methods for scaling test-time computation have shown limited gains in math reasoning, often due to policy LLM or reward model limitations.   {\sysname} addresses this by iteratively evolving the policy LLM and reward model, achieving System 2 mathematical reasoning performance comparable to OpenAI o1~\citep{o1}.
   
   
   
   
   
   \noindent\textbf{Reward Models} are crucial for effective System 2 reasoning but are challenging to obtain. Recent works include LLM-as-a-Judge for verification ~\citep{llmjudge,rstar}  and specialized reward models like Outcome Reward Model~\citep{qwen2.5,yu2023outcome} and Process Reward Model (PRM)~\citep{lightman2024lets}. While PRMs offer promising dense, step-level reward signals for 
complex reasoning~\citep{luo2024improve,mathshepherd},  collecting step-level annotations remains an obstacle. While ~\cite{kang2024mindstar, wang2024qimprovingmultistepreasoning} rely on costly human-annotated datasets like PRM800k~\citep{lightman2024lets},
   recent approaches~\citep{mathshepherd,luo2024improve} explore automated annotation via Monte Carlo Sampling or MCTS. However,  they struggle to generate precise reward scores, which limits performance gains. {\sysname} introduces a novel process preference reward (PPM) that eliminates the need for accurate step-level reward score annotation. 
   
  
   
   
 













%\noindent\textbf{Self-Improvement}.   
