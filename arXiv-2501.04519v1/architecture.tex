\section{System Architecture and Implementation}

\subsection{Core Technologies}

\sysname{} is built on a modern technology stack:

\begin{itemize}
    \item \textbf{Language}: Rust (Edition 2021) for performance and safety
    \item \textbf{Web Framework}: Actix-web 4.4 for async web services
    \item \textbf{Database}: PostgreSQL 14.2+ with JSON support
    \item \textbf{Caching}: Redis 6.2+ for session management
    \item \textbf{Runtime}: Tokio 1.35+ for async I/O
\end{itemize}

\subsection{File Structure}

The project follows a well-organized directory structure:

\begin{lstlisting}[language=bash]
lotabots/
├── services/           # Microservices
│   ├── api-gateway/
│   ├── auth/
│   ├── attestation/
│   ├── document/
│   └── resource-management/
├── shared/            # Shared libraries
│   ├── config/
│   ├── db/
│   ├── models/
│   └── utils/
├── infrastructure/    # IaC
├── scripts/          # Development scripts
├── docs/             # Documentation
├── tests/            # Integration tests
└── tools/            # Development tools
\end{lstlisting}

\subsection{Service Architecture}

Each microservice follows a consistent architecture:

\begin{itemize}
    \item \textbf{Handlers}: Process incoming HTTP requests
    \item \textbf{Models}: Define data structures and validation
    \item \textbf{Repository}: Handle database operations
    \item \textbf{Services}: Implement business logic
    \item \textbf{Config}: Manage service configuration
\end{itemize}

\subsection{Security Implementation}

Security measures are implemented at multiple levels:

\begin{itemize}
    \item \textbf{Authentication}: JWT-based with refresh tokens
    \item \textbf{Password Security}: bcrypt with configurable work factor
    \item \textbf{API Security}: Rate limiting, CORS, CSP headers
    \item \textbf{Secrets}: HashiCorp Vault or AWS Secrets Manager
    \item \textbf{Network}: TLS 1.3, mTLS for service communication
\end{itemize}

\subsection{Deployment Architecture}

The system is designed for cloud-native deployment:

\begin{itemize}
    \item \textbf{Containerization}: Multi-stage Docker builds
    \item \textbf{Orchestration}: Kubernetes with Helm charts
    \item \textbf{Infrastructure}: AWS (EKS, RDS, ElastiCache)
    \item \textbf{CI/CD}: GitHub Actions with security scans
    \item \textbf{Monitoring}: Prometheus, Grafana, ELK Stack
\end{itemize}

\subsection{Development Workflow}

The development process follows best practices:

\begin{itemize}
    \item \textbf{Code Quality}:
    \begin{itemize}
        \item cargo fmt for consistent formatting
        \item clippy for static analysis
        \item cargo audit for security checks
    \end{itemize}
    \item \textbf{Testing}:
    \begin{itemize}
        \item Unit tests with cargo test
        \item Integration tests with test containers
        \item End-to-end tests with real services
    \end{itemize}
    \item \textbf{Documentation}:
    \begin{itemize}
        \item OpenAPI specifications
        \item Inline documentation
        \item Architecture decision records
    \end{itemize}
    \item \textbf{Version Control}:
    \begin{itemize}
        \item Git with conventional commits
        \item Pull request reviews
        \item Automated CI checks
    \end{itemize}
\end{itemize}
